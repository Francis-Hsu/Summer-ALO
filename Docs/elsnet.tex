\documentclass[11pt]{article}
\usepackage{fullpage,titling}
\usepackage{mathtools,amssymb,amsthm}
\usepackage{bm}
\usepackage{tikz}
\usepackage{hyperref}
\usepackage{array}
\usepackage{float}
\usepackage{subcaption}
\usepackage{lstautogobble}
\usepackage[T1]{fontenc}
\usepackage{newpxtext,newpxmath}
\usepackage[activate={true,nocompatibility},final,tracking=true, kerning=true, spacing=true, factor=1100, stretch=10, shrink=10]{microtype}

\newcommand{\bu}{\bm{u}}
\newcommand{\bx}{\bm{x}}
\newcommand{\by}{\bm{y}}
\newcommand{\bz}{\bm{z}}
\newcommand{\bA}{\bm{A}}
\newcommand{\bD}{\bm{D}}
\newcommand{\bH}{\bm{H}}
\newcommand{\bI}{\bm{I}}
\newcommand{\bJ}{\bm{J}}
\newcommand{\bX}{\bm{X}}
\newcommand{\bY}{\bm{Y}}
\newcommand{\bbeta}{\bm{\beta}}
\newcommand{\btheta}{\bm{\theta}}

\newcommand{\bbr}{\mathbb{R}} 
\newcommand{\bbq}{\mathbb{Q}}
\newcommand{\bbn}{\mathbb{N}}

\newcommand{\semicol}{\nobreak\mskip2mu\mathpunct{}\nonscript\mkern-\thinmuskip{;}\mskip6muplus1mu\relax}

\DeclareMathOperator{\sign}{sgn}
\DeclareMathOperator{\prox}{prox}
\DeclareMathOperator{\diag}{diag}
\DeclareMathOperator{\bprox}{\mathbf{prox}}
\DeclareMathOperator*{\argmin}{arg\,min}

\newcommand{\refthm}[2]{#1~#2}

\title{Notes on Approximate Leave-One-Out for Elastic Net}
\author{Linyun He \and Wanchao Qin \and Peng Xu \and Yuze Zhou}

\begin{document}
\maketitle

\section{ALO for Elastic Net, Approximation in the Primal Domain}
Recall the objective function for the elastic net problem:
	\begin{equation}
	\min_{\bbeta}\frac{1}{2}\sum_{j=1}^{n}(\bx_j^\top\bbeta-y_j)^2+\lambda\left(\alpha\|\bbeta\|_1+\frac{1-\alpha}{2}\|\bbeta\|_2^2\right).
	\end{equation}
Let \(A=\{i:\beta_i\not\in K,i=1,\dotsc,p\}\) be the active set, we have \[\dot{\ell}(\bx_j^\top\bbeta\semicol y_j)=\bx_j^\top\bbeta-y_j,\qquad\ddot{\ell}(\bx_j^\top\bbeta\semicol y_j)=1,\qquad\nabla^2 R(\hat{\bbeta}_A)=(1-\alpha)\lambda\bI_{A,A}.\] Thus, \refthm{Eqn.}{31} reduces to
	\begin{equation}
	\bH=\bX_{\cdot,A}\left[\bX_{\cdot,A}^\top\bX_{\cdot,A}+\left(1-\alpha\right)\lambda\bI_{A,A}\right]^{-1}\bX_{\cdot,A}^\top.
	\end{equation}
By augmenting \(\bX\) with an extra column of \(1\)s, adding the intercept back to the model is straightforward, as \refthm{Eqn.}{31} now becomes 
	\begin{equation}
	\bm{H}=\left[\bm{1}_n,\bm{X}_{\cdot, A}\right]\left\{\left[\bm{1}_n,\bm{X}_{\cdot, A}\right]^\top\bm{D}\left[\bm{1}_n,\bm{X}_{\cdot, A}\right]+\nabla^2R\left(\hat{\beta}_0,\hat{\bm{\beta}}_A\right)\right\}^{-1}\left[\bm{1}_n,\bm{X}_{\cdot, A}\right]^\top,
	\end{equation}
where \[\bm{D}=\diag\left[\ddot{\ell}\left(\hat{\beta}_0+\bm{x}_j^\top\hat{\bm{\beta}};y_j\right)\right]_{j\in A}=\bI_{A,A},\qquad\nabla^2R\left(\hat{\beta}_0,\hat{\bm{\beta}}_A\right)=\begin{bmatrix}
0 & 0 & \dots & 0 \\
0 & (1-\alpha)\lambda & \dots & 0\\
\vdots & \vdots & \ddots & \vdots\\
0 & 0 & \dots & (1-\alpha)\lambda \\
\end{bmatrix}.\] Then, the ALO can be computed as
	\begin{equation}
	\begin{bmatrix}
	1 & \bm{x}_i^\top\end{bmatrix}
	\begin{bmatrix}
	\tilde{\beta}_0^{\setminus i} \\
	\tilde{\bm{\beta}}^{\setminus i}\end{bmatrix}=(\hat{\beta}_0+\bm{x}_i^\top\hat{\bm{\beta}})+\frac{\bH_{ii}}{1-\bH_{ii}\ddot{\ell}\left(\hat{\beta}_0+\bm{x}_i^\top\hat{\bm{\beta}};y_i\right)}\dot{\ell}\left(\hat{\beta}_0+\bm{x}_i^\top\hat{\bm{\beta}};y_i\right)
	\end{equation}

\section{ALO for Elastic Net, Approximation in the Dual Domain}
The original problem for elastic net is to solve for $\hat{\bbeta}$ such that: 
\begin{equation}
\hat{\bbeta}=\argmin\limits_{\bbeta}\left(\frac{1}{2}\|\by-\bX\bbeta\|_{2}^{2} + \lambda_{1}\|\bbeta\|_{1}+\lambda_{2}\|\bbeta\|_{2}^{2}\right)
\end{equation} 
By adding the Lagrangian, we get the formulation of $L$: 
	\begin{equation}
	L = \frac{1}{2}\|\by-\bz\|_{2}^{2} + \lambda_{1}\|\bbeta\|_{1}+\lambda_{2}\|\bbeta\|_{2}^{2}+u^{\top}(\bz-\bX\bbeta).
	\end{equation}
The original problem is solving the primal of the Lagrangian such that $p^{*} = \min\limits_{\bbeta,\bz}\max\limits_{\bu}L$ and the dual formulation $d^{*} = \max\limits_{\bu}\min\limits_{\bbeta,\bz}L$, to minimize over $\bz$: \[\frac{\partial L}{\partial \bz} = \bz -\by +\bu = \bm{0}\implies \by = \bu + \bz.\] Since $\bbeta$ is penalized element-wisely, we can minimize over $\bbeta$ by minimizing over each $\bbeta_{i}$, that is, we have to minimize $\lambda_{1}|\beta_{i}| + \lambda_{2}\beta_{i}^{2} - \bu^{\top}\bX_{i}\bbeta$ for each dimension of $\bbeta$, where $\bX_{i}$ denotes the $i$th column of $\bX$, therefore: \[\min\limits_{\bbeta}\left(\lambda_{1}|\beta_{i}| + \lambda_{2}\beta_{i}^{2} - \bu^{\top}\bX_{i}\bbeta\right)=\begin{dcases}
0 & |\bu^{\top}\bX_{i}| \leq \lambda_{1},\\
-\frac{(\lambda_{1}-|\bu^{\top}\bX_{i}|)^{2}}{4\lambda_{2}} & |\bu^{\top}\bX_{i}| > \lambda_{1}.
\end{dcases}\] By taking all the above to the Lagrangian, we obtain the dual problem $d^{*}$ as: 
	\begin{equation}
	d^{*} = \min\limits_{\bu}\frac{1}{2}\|\by-\bu\|_{2}^{2} + \sum_{j: |\bX_{j}^{\top}\bu| > \lambda_{1}}\frac{(\lambda_{1}-|\bu^{\top}\bX_{i}|)^{2}}{4\lambda_{2}}.
	\end{equation}
The minimizer $\hat{\bu}$ could also be obtained from the dual problem through a proximal approach: \[\hat{\bu} = \bprox_{R}(y),\qquad\quad R(\bu) = \sum_{j:|\bX_{j}^{\top}\bu| > \lambda_{1}}\frac{(\lambda_{1}-|\bu^{\top}\bX_{i}|)^{2}}{4\lambda_{2}}.\]

By replacing the full data problem $\by$ with $\by_{\alpha} = \by + (y_{i}^{\setminus i}-y_{i})e_{i}$, where $y_{i}^{\setminus i}$ is the true LOO estimator and $e_{i}$ is the $i$-th standard vector, and let $\bu^{\setminus i} = \bprox_{R}(\by_{\alpha})$, we have:
\begin{align*}
0 &= e_{i}^{\top}\bu^{\setminus i}\\
& = e_{i}^{\top}\bprox_{R}(\by_{\alpha})\\
& \approx e_{i}^{\top}[\bprox_{R}(\by)+\bJ_{R}(\by)(\by_{\alpha}-\by)]\\
& \approx \hat{u}_{i} + \bJ_{ii}(y_{i}^{\setminus i}-y_{i}).
\end{align*} Here $\bJ_{R}(\by)$ denotes the Jacobian matrix of the proximal operator at $\by$, thus the ALO estimator $\tilde{y}_{i}$ is obtained as 
	\begin{equation}
	\tilde{y}_{i} = y_{i} - \frac{\hat{u}_{i}}{\bJ_{ii}}.
	\end{equation}
The Jacobian could locally be obtained as:
	\begin{equation}
	\bJ_{R}(\by)= (\bI+\nabla^{2}R(\bprox_{R}(\by)))^{-1}= (\bI + \nabla^{2}R(\hat{\bu}))^{-1}= \left(\bI + \frac{1}{2\lambda_{2}}\bX_{E}\bX_{E}^{\top}\right)^{-1}
	\end{equation}
for $E = \{j:|\bX_{j}^{\top}\bu|>\lambda_{1}\}$.

\section{ALO for Elastic Net, Approximation with Proximal Formulation}
For the elastic net problem, the proximal mapping is known to be
	\begin{equation}
	\bprox_R\left(\bz\right)=\gamma\sign(\bz)\odot(|\bz|-\lambda\bm{1}_p)_+,\qquad\gamma=\frac{1}{1+(1-\alpha)\lambda}.
	\end{equation}
Let \(E\) be the active set, if \(z_i\in E\), then \[\frac{\partial}{\partial z_i}\gamma\sign(z_i)(|z_i|-\lambda)_+=\gamma.\] Plug in \(\bz=\hat{\bbeta}-\sum_{j=1}^{n}\dot{\ell}(\bx_j^\top\hat{\bbeta}\semicol y_j)\bx_j\), \refthm{Eqn.}{46} thus reduce to 
	\begin{equation}
	\bH=\gamma\bX_{\cdot,E}\left[\gamma\bX_{\cdot,E}^\top\bX_{\cdot,E}+\left(1-\gamma\right)\bI_{E,E}\right]^{-1}\bX_{\cdot,E}^\top.
	\end{equation}
Bringing back the intercept term is straightforward as well. Noted that \[\begin{bmatrix}
\hat{\bbeta}_0^{\setminus i} \\
\hat{\bm{\bbeta}}^{\setminus i}\end{bmatrix}=
\bprox_{R}\left(\bz\right),\qquad\bz=\begin{bmatrix}
\hat{\bbeta}_0^{\setminus i} \\
\hat{\bm{\bbeta}}^{\setminus i}\end{bmatrix}-
\sum_{j\neq i}\begin{bmatrix}
1 \\
\bm{x}_j\end{bmatrix}
\dot{\ell}\left(\hat{\bbeta}_0^{\setminus i}+\bm{x}_j^\top\hat{\bm{\bbeta}}^{\setminus i};y_j\right).\] Hence, from the first-order condition \(\sum_{j\neq i}\dot{\ell}\left(\hat{\bbeta}_0^{\setminus i}+\bm{x}_j^\top\hat{\bm{\bbeta}}^{\setminus i};y_j\right)=0\), we can derive that 
	\begin{equation}
	\bm{J}_{E,E}=
	\left[\bm{J}(\bm{u})\right]_{E,E}=
	\begin{bmatrix}
	1 & 0 & 0 & \dots & 0\\
	0 & (1-\alpha)\lambda & 0 & \dots & 0\\
	0 & 0 & (1-\alpha)\lambda & \dots & 0\\
	\vdots & \vdots & \vdots & \ddots & \vdots\\
	0 & 0 & 0 & 0 & (1-\alpha)\lambda\\
	\end{bmatrix}^{-1}.
	\end{equation}
The ALO formula is then immediate by \refthm{Thm.}{5.1}.

\section{ALO for LASSO, with Intercept through Generalized LASSO}
For the generalized LASSO:
	\begin{equation}
	\min\limits_{\bbeta}\frac{1}{2}\|\by-\bX\bbeta\|+\lambda\|\bD\bbeta\|_{1},
	\end{equation}
the dual problem can be derived as:
	\begin{equation}
	\min\limits_{\bu}\frac{1}{2}\|\by-\btheta\|_{2}^{2},\qquad\theta\in \{\bX^{\top}\btheta = \bD^{\top}\bu, \|\bu\|_{\infty} \leq \lambda\}.
	\end{equation}
The dual problem could be written in a proximal approach, such that: \[\hat{\bu} = \bprox_{R}(\by),\qquad R(\bu) =\begin{dcases}
0 & \btheta \in \{\bX^{\top}\btheta = \bD^{\top}\bu, \|\bu\|_{\infty} \leq \lambda\},\\
\infty & \text{otherwise.}
\end{dcases}\] Denote $\bJ$ as the Jacobian of the proximal operator at the full data problem $\by$, then the ALO estimator could be obtained as: 
	\begin{equation}
	\by^{\setminus i} = \by_{i} - \frac{\hat{\bu}_{i}}{\bJ_{ii}}.
	\end{equation}
For the case of LASSO with an intercept, we could expand the $\bX$ with a column of ones in the first column, expand $\bbeta$ with another dimension and choose $\bD = [\bm{0}, \bI]$. Let \(E\coloneqq\{j:|\bX_{j}^{\top}\theta| = \lambda \}\) denote the active set. The Jacobian is locally given as the projection onto the orthogonal complement of the span of $\bX_{E}$ and the vector of ones. Further denote $\tilde{\bX}_{E} = [\textbf{1}, \bX_{E}]$, then the Jacobian is given as $\bI - \tilde{\bX_{E}}(\tilde{\bX}_{E}^{\top}\tilde{\bX}_{E})\tilde{\bX}_{E}^{\top}$.

\section{ALO for Elastic Net, without Penalty on Intercept through Generalized LASSO}
Without penalty on intercept, the elastic net problem can be written as:
\begin{align*}
\begin{bmatrix}
\hat{\bbeta}_0 \\
\hat{\bbeta}
\end{bmatrix} &= \argmin\frac{1}{2} \left\|\by - \beta_0 - \bX\bbeta\right\|_2^2 + \lambda_1 \|\bbeta\|_1 + \lambda_2 \|\bbeta\|_2^2 \\
&= \argmin\frac{1}{2} 
\begin{bmatrix}
\beta_0 \\
\bbeta
\end{bmatrix}^\top \left(
\begin{bmatrix}
1 & \bX
\end{bmatrix}^\top
\begin{bmatrix}
1 & \bX
\end{bmatrix} + 
\lambda_2 \diag(0; \bm{1}_p)
\right)
\begin{bmatrix} 
\beta_0 \\
\bbeta
\end{bmatrix} - \by^\top
\begin{bmatrix}
1 & \bX
\end{bmatrix} 
\begin{bmatrix}
\beta_0 \\
\bbeta
\end{bmatrix}   + \lambda_1 \|\bbeta\|_1
\end{align*}
where we assume that the size of $\bX$ is $n \times p$. In the mean time, note the LASSO problem (also without penalty on intercept) is:
\begin{align*}
\begin{bmatrix}
\hat{\bbeta}_0 \\
\hat{\bbeta}
\end{bmatrix} &= \argmin\frac{1}{2} \left\|\by - \beta_0 - \bX\bbeta\right\|_2^2 + \lambda_1 \|\bbeta\|_1 \\
&= \argmin\frac{1}{2} 
\begin{bmatrix}
\beta_0 \\
\bbeta
\end{bmatrix}^\top 
\begin{bmatrix}
1 & X
\end{bmatrix}^\top
\begin{bmatrix}
1 & X
\end{bmatrix} 
\begin{bmatrix} 
\beta_0 \\
\bbeta
\end{bmatrix} - \by^\top
\begin{bmatrix}
1 & X
\end{bmatrix} 
\begin{bmatrix}
\beta_0 \\
\bbeta
\end{bmatrix} +  \lambda_1 \|\bbeta\|_1
\end{align*}
Thus we can add some ``observations'' to the data and let
$$\by^\ast = \begin{bmatrix}
\by \\
\bm0_p
\end{bmatrix},\qquad \bX^\ast =
\begin{bmatrix}
\bX \\
\sqrt{\lambda_2} \bm{I}_p
\end{bmatrix}, $$
then the elastic net becomes
	\begin{equation}
		\begin{aligned}
		\begin{bmatrix}
		\hat{\bbeta}_0 \\
		\hat{\bbeta}
		\end{bmatrix} &= \argmin\frac{1}{2} \left\|\by^\ast - \bbeta_0
		\begin{bmatrix}
		\bm{1}_n \\
		\bm{0}_p
		\end{bmatrix} - \bX^\ast\bbeta\right\|_2^2 + \lambda_1 \|\bbeta\|_1  \\
		&= \argmin\frac{1}{2} \left\|
		\begin{bmatrix}
		\by \\
		\bm0_p
		\end{bmatrix}
		- 
		\begin{bmatrix}
		\bm{1}_n & \bX \\
		\bm{0}_p &\sqrt{\lambda_2} \bm{I}_p
		\end{bmatrix}  
		\begin{bmatrix}
		\bbeta_0 \\
		\bbeta
		\end{bmatrix} 
		\right \|_2^2 + \lambda_1 \|\bbeta\|_1,
		\end{aligned}
	\end{equation}
which is a special case of the general LASSO.

\section{Usage of ALO formulae with \texttt{glmnet} package}
The \verb|glmnet| package scales the elastic net loss function by a factor of \(1/n\), so the ALO formulae must be adjusted accordingly, e.g. for the proximal case, we instead have: \[\tilde{\by}_j^{\setminus i}=\hat{\by}_j+\frac{\bH_{ii}(\hat{\by}_j-\by_j)}{n-\bH_{ii}},\qquad\bH=\gamma\bX_{\cdot,E}\left[\frac{\gamma}{n}\bX_{\cdot,E}^\top\bX_{\cdot,E}+\left(1-\gamma\right)\bI_{E,E}\right]^{-1}\bX_{\cdot,E}^\top.\]

Furthermore, \verb|glmnet| implicitly ``standardizes \(y\) to have unit variance before computing its \(\lambda\) sequence (and then unstandardizes the resulting coefficients)'' (\emph{cf.} [\href{https://web.stanford.edu/~hastie/glmnet/glmnet_alpha.html}{Glmnet Vignette}]). So to get comparable results, it is necessary to rescale \(\by\) by the MLE \(\hat{\sigma}_y\) before fitting the model. \autoref{fig:elsnet} shows the comparison of the ALO and LOO for different \(\alpha\)s. Without standardizing \(\by\) first, a growing discrepancy between the two curves can be observed as \(\alpha\to0\).
	\begin{figure}[!htbp]
	\centering
	\begin{subfigure}[b]{\textwidth}
		\includegraphics[width=\textwidth]{elsnet1.png}
		\caption{With standardization on \(\by\).}
	\end{subfigure}
	\begin{subfigure}[b]{\textwidth}
		\includegraphics[width=\textwidth]{elsnet2.png}
		\caption{Without standardization on \(\by\)}
	\end{subfigure}
	\caption{ALO vs. LOO for elastic net with intercept, misspecification example.\label{fig:elsnet}}
	\end{figure}
\end{document}